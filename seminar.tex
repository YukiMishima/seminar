\documentclass[12pt]{ltjsarticle}

\usepackage{amsmath}  %文字をイタリックにしない
\usepackage{graphicx}

\begin{document}
\begin{titlepage}
\title{金属表面の反応性を決定する指標}
\author{まるまる研 B4 \\ なまえ}
\date{2018/12/19}
\maketitle
%\thispagestyle{empty}
\tableofcontents
\end{titlepage}

\section{はじめに}
近年、遷移金属を用いた触媒は広く使用されており、そういった触媒材料を探索する際には
遷移金属表面の化学的反応性が重要になる。
また、表面化学の分野においても金属表面の反応性を決定づけている物理的性質の理解については
長年興味がもたれており、分子-金属表面間での相互作用の理論的解釈について多くの研究が
なされてきた。 \\

\subsection{先行研究}

過去に考案されてきた代表的な金属の反応性指標の2つは以下のとおりである。
\begin{itemize}
 \item フェルミ準位周辺の局所状態密度
 \item d空孔の数 $\text{N}_\text{h}$
\end{itemize}

これらの指標について簡単に説明する。

\subsubsection{フェルミ準位周辺の局所状態密度}
Peter,Hammanはフェルミ準位における局所状態密度(Local Density of States at Fermi level.
以下、LDOS($\text{E}_\text{F}$))が金属表面の反応性を決定づけていると提唱した。
これは、LDOS($\text{E}_\text{F}$)が大きければ、フェルミ準位から反応物の非占有準位への
電荷移動が起こりやすくなるため、LDOS($\text{E}_\text{F}$)が大きな金属ほど反応性が高いという
考え方である。\cite{PeterJ.Feibelman1984}

\subsubsection{d空孔の数$\text{N}_\text{h}$}
金属dバンドの上位の電子がsバンドに移ると空孔ができる。これをd空孔(d-holes)と呼び、
その数を$\text{N}_\text{h}$と表現する。
遷移金属のd空孔にs電子が戻るというd-s遷移(例えば5d空孔に6s軌道から遷移)により、パウリ反発
(分子軌道と金属d軌道とが重なることによるエネルギーの上昇)が減少する。\cite{MORIKAWA2006}
これにより金属表面-分子間結合の形成が促されるため、結果的にd空孔の数$\text{N}_\text{h}$が多いほどその
金属の反応性が高くなるとHarris,Anderssonは結論づけた。\cite{J.Harris1985}

\subsection{先行研究の破綻}
以上に述べた二つの指標は金属表面のおおよその反応性を決定づけることができるものの
すべての金属の反応性を満足するわけではなかった。\\
そこで今回は、あらゆる金属表面の反応性を満たすような別の指標を提示する。\cite{Science1995}

\section{吸着エネルギーの計算}
金属の反応性を見るため、いくつかの金属表面上において水素の解離吸着反応に伴う
エネルギー変化を計算し比較した。\\
実験に用いた金属は以下の4つである。
\begin{itemize}
 \item Al :[Ne](3s)$^\text{2}$(3p)$^\text{1}$
 \item Cu :[Ar](3d)$^\text{10}$(4s)$^\text{1}$
 \item Pt :[Xe](4f)$^\text{14}$(5d)$^\text{9}$(6s)$^\text{1}$
 \item $\text{Cu}_\text{3}$Pt
\end{itemize}
この4種類の金属が選出された理由はそれぞれ、dバンドを持たない金属、LDOS($\text{E}_\text{F}$)が
大きい金属、LDOS($\text{E}_\text{F}$)が小さい金属、そしてd空孔を持たない金属を表現するためである。

\subsection{計算方法}

\begin{figure}[hbtp]
\begin{minipage}{.5\textwidth}
    各金属表面上での水素の解離吸着反応に伴うエネルギー変化を計算するにあたり、スラブ
    モデルを用いてDFT計算を行った。
    Al、Cu、Ptはfcc構造をとり格子定数はそれぞれ、3.96$\mbox{\AA}$、3.57$\mbox{\AA}$、
    3.93$\mbox{\AA}$である。$\text{Cu}_\text{3}$Ptは$\text{Cu}_\text{3}$Auに似た、
    fccライクな構造(図\ref{fig:Cu3Au})をしており格子定数は3.68$\mbox{\AA}$である。
\end{minipage}
\hfill
\begin{minipage}{.45\textwidth}
    \begin{center}
     \includegraphics[width=4.5cm]{figure/Cu3Au.jpg}
    \end{center}
    \caption{Cu$_\text{3}$Auの構造}
    \label{fig:Cu3Au}
\end{minipage}
\end{figure}
表面のスラブモデルはすべて、金属(111)表面6層+吸着真空層5層からなる
2×2表面ユニットセルを持つスーパーセルを用いた。
50Ry(約680eV)のカットオフエネルギーを持つ平面波基底を用い、すべての金属に
おいてノルム保存型擬ポテンシャルを用いた。\footnote{AlにはBHS型、CuとPtに対してはTroullier-Martins型のものを用いた。}

\subsection{計算結果}
各金属の(111)面上での$\text{H}_\text{2}$のエネルギー計算結果は図\ref{fig:potential}Aのよう
になった。\\
\begin{figure}[hbtp]
    \begin{center}
     \includegraphics[width=8cm]{potential.png}
    \end{center}
    \caption{各金属上での水素の吸着エネルギー}
    \label{fig:potential}
\end{figure}
水素分子(あるいは水素原子)から金属表面までの距離Zを横軸にとり、縦軸がエネルギーである。
縦軸は各金属の洗浄表面の全エネルギーと孤立水素分子のエネルギーの和を0としている。
水素吸着の定量的な議論のため、各金属とおして結合長bと高さZは同じ値を用いて計算を行った。
\footnote{(b,z)=(0.78$\mbox{\AA}$,3.0$\mbox{\AA}$),(0.8$\mbox{\AA}$,2.0$\mbox{\AA}$),(0.9$\mbox{\AA}$,1.75$\mbox{\AA}$),(1.2$\mbox{\AA}$,1.5$\mbox{\AA}$),(2.0$\mbox{\AA}$,1.25$\mbox{\AA}$),
($\sqrt{2/3}a,0.9\mbox{\AA}$),($\sqrt{2/3}a,0.75\mbox{\AA}$) ただし、aはそれぞれの格子長。}
Al、Cu、$\text{Cu}_\text{3}$Pt内のCuサイトでの解離吸着は大きなエネルギー障壁が確認できる
一方で、Ptや$\text{Cu}_\text{3}$Pt内のPtサイトには障壁はない。

我々の選択した反応経路の妥当性を調べるため、水素分子の結合長bと距離Zをリラックスさせた場合の
吸着エネルギーの計算をCu表面について行った。計算結果は図\ref{fig:potential}の点線で示されている
とおりである。エネルギー障壁は若干小さくなるものの、傾向は変わらないことが確認できる。

また、計算したポテンシャルのこれらの特徴、すなわち、CuとAl表面では解離吸着に大きな障壁を伴い、
Ptと$\text{Cu}_\text{3}$Ptでは自発的に反応が進むという傾向は実験の結果と同じであった。

\section{考察}

金属表面の反応性の議論を各金属の状態密度(Density of States)について考えることから始める。
図\ref{fig:dosmetal}はCu単体でのDOS, Pt単体でのDOS, その間に$\text{Cu}_\text{3}$Ptの
DOSを示している。実線は各原子の状態密度を表し、点線はバルク状態での状態密度を表している。
\begin{figure}[hbtp]
    \begin{center}
     \includegraphics[width=8cm]{figure/DosOfMetal.png}
    \end{center}
    \caption{各金属の状態密度}
    \label{fig:dosmetal}
\end{figure}
$\text{Cu}_\text{3}$Ptではdバンドの頂点がFermi準位と交差していることに注目されたい。
これは、$\text{Cu}_\text{3}$Ptのdバンドがほどんど占有されておりd空孔がないことを示している。
また$\text{Cu}_\text{3}$Pt中のPtサイトのLDOS($\text{E}_\text{F}$)はPt単体より小さいことが分かる。

従来の指標、LDOS($\text{E}_\text{F}$)や$\text{N}_\text{h}$を用いると以上の事実は
$\text{Cu}_\text{3}$PtのPtサイトはPt単体よりも表面の反応性は低いという結論に帰着する。
しかしながら図\ref{fig:potential}によると$\text{Cu}_\text{3}$PtのPtサイトはPt単体と
同じ程度の吸着エネルギー(反応性)になっている。したがってこれはLDOS($\text{E}_\text{F}$)や
$\text{N}_\text{h}$では説明できないケースということになる。

\subsection{金属表面での吸着エネルギーの定式化}
吸着エネルギーを遷移金属のDOSと関連付けるため、Kohn-Sham方程式により求まる軌道エネルギー
を用いる。ここで、以下の近似を導入する。\\
2つの異なる金属表面上での吸着質のエネルギーを比較する際、吸着質に近い場所の電子密度と
1電子ポテンシャルを固定することができ、異なる系に対して同じ密度とポテンシャルを
用いることができる。同様に、吸着質に近い場所より外側の部分に対しても、密度とポテンシャル
を固定でき、吸着質の存在とは独立に金属のみとして扱うことができる。
この近似を用いると異なる2つの金属表面における吸着エネルギーの差は式(\ref{eabs})により表せる。
\begin{equation}
    \label{eabs}
    \delta E_{abs} = \int_{}^{E_F} \varepsilon n_{abs}(\varepsilon) d\varepsilon + 
    \delta E_{es}
\end{equation}
1項目は、吸着に関する軌道におけるエネルギーの差であり、
2項目は、吸着質の静電エネルギーの差である。変分原理のために、エネルギー差の誤差は密度と
ポテンシャルを固定していることによる誤差の二次オーダーである。
式(\ref{eabs})より、大きな電荷移動がなく静電エネルギーの差が重要でない場合は障壁の大きさの
違い、すなわち1項目の軌道エネルギーの差から金属の反応性を見積もることができる。

\subsubsection{d電子を持たない金属表面の反応性}
式(\ref{eabs})を用いて吸着エネルギーの差における定性的な理解を試みよう。
まずは、d電子を持たず、s,p電子のみを持つ金属表面と水素分子の相互作用を考える。
\begin{figure}[hbtp]
    \begin{center}
     \includegraphics[width=11cm]{figure/interactedDOS.png}
    \end{center}
    \caption{分子と相互作用した金属の状態密度}
    \label{fig:interactedDOS}
\end{figure}

水素原子が金属表面のs,p電子とだけ相互作用した場合、
水素の結合性軌道と反結合性軌道の概略図は図\ref{fig:interactedDOS}Bのようになる。
反結合性軌道はFermi準位よりもわずかに上、結合性軌道はFermi準位よりも約7eV下に位置している。
\begin{figure}[hbtp]
    \begin{center}
     \includegraphics[width=12cm]{figure/dinteractedDOS.png}
    \end{center}
    \caption{分子-金属d相互作用した金属の状態密度}
    \label{fig:dinteractedDOS}
\end{figure}

今回の実験では、図\ref{fig:dinteractedDOS}AのAl(111)がd電子を持たない金属表面の例である。
図\ref{fig:potential}のAl(111)のエネルギー曲線より、自由電子が解離吸着に対して大きな障壁を
もたらしていることを示している。\cite{Mysyrowicz1993}

\subsubsection{d電子をもつ金属の反応性}
次にAl以外の金属、つまりd電子を持つ金属表面の反応性について考える。\\
d電子相互作用によるエネルギーの寄与は式(\ref{eabs})の1項目、軌道エネルギーの差から
見積もることができる。
図\ref{fig:interactedDOS}B-Dはd準位の導入による効果を表している。
Cuのdバンドと水素の分子軌道の相互作用の結果、反結合性軌道のピークは分子準位の中心と
dバンド中心の上に、結合性軌道のピークは分子準位の中心とdバンド中心の下に位置したDOSが
得られている。
図\ref{fig:potential}より、dバンドとの相互作用がエネルギー障壁を下げるのに
寄与していることが分かる。

\subsection{dバンドと水素分子軌道の混成によるエネルギー変化の定式化}
この相互作用の大きさ、すなわちdバンドと水素分子軌道の混成によるエネルギーのパラメータは
摂動の式(\ref{deltaepsilon})により与えられる。
\begin{equation}
    \label{deltaepsilon}
    \Delta \varepsilon ~ \frac{V^2 }{| \varepsilon _d - \varepsilon _ \sigma |}, \\
\end{equation}
dバンドとの相互作用の影響は3つのパラメータに依存する。
\begin{enumerate}
  \item dバンドセンターと、分子の結合性準位と反結合性準位の位置の差
  $| \varepsilon _d - \varepsilon _ \sigma |$
  \item 分子軌道と金属dバンドとの間のハミルトニアン行列要素V
  \item 反結合性$\sigma_\text{g}$-d準位の占有度
\end{enumerate}

\subsubsection{Cuで見る分子ー金属d相互作用の効果}
図\ref{fig:interactedDOS}CのCuを例にとってみると、反結合性$\text{\sigma}_\text{u}^\text{*}$
準位がほとんど空なので引力的である一方、$\text{\sigma}_\text{g}$-d相互作用では反結合性のものが
占有されているためパウリ反発により不安定化する。トータルでは、分子-金属d相互作用によってAlよりも
Cuの方が障壁が少し低いという結果になる。

\subsubsection{そのほかの金属で見る分子-d相互作用}
図\ref{fig:dinteractedDOS}より反結合性$\sigma _\text{u} ^* $-d 準位はどの金属においても
Fermi準位の上に位置していることからこの準位は占有されていないことが分かる。
したがって、$ \sigma _\text{u} ^*$-d相互作用は常に安定化に寄与している。
Cuと$\text{Cu}_\text{3}$Ptでは結合強度(V)とdバンドの位置はどちらもほとんど同じであるため、
$ \sigma _\text{u}^*$-d相互作用の大きさは同程度であると考えられる。
PtとCuではdバンドの位置はほとんど同じ(図\ref{fig:dosmetal})であるが、
Ptの方がより結合強度が大きいためPtはCuよりも$ \sigma _\text{u}^*$-d相互作用が大きい。
結合強度の違いは相互作用に用いられるd軌道がCuは3d軌道であるのに対してPtはよりバンドが
大きい5d軌道であることに起因する。

$\sigma _\text{g}$-d相互作用は、表面の金属によってかなり大きさが変わる。
図\ref{fig:dinteractedDOS}からわかるように、反結合性$\sigma _\text{g}$-dピークは
Cuではちょうどdバンドの上あたりにあり
$\text{Cu}_\text{3}$Ptではフェルミ準位のあたり、Ptではフェルミ準位の上あたりにシフトしていく。

\section{反応性の新たな指標}
より定量的な議論のため、反応性の指標を導入する。dバンドの相互作用によるエネルギー差は
式(\ref{deltaepsilon})を用いて式(\ref{deltaets})と表せる。
\begin{eqnarray}
    \label{deltaets}
    \delta E_{d-hyb} = -2 \frac{ V^2 }{\varepsilon _{\sigma_u^*} - \varepsilon _d}
                    -2(1-f)\frac{V^2}{\varepsilon _d - \varepsilon _{\sigma_g}}
                    + \alpha V^2
\end{eqnarray}

1項目は$\sigma _\text{u}^* -\text{d}$相互作用によるエネルギー変化、
2項目は$\sigma _\text{g}-\text{d}$相互作用によるエネルギー変化である。
係数の2はどちらも、上下のスピンに対応している。2項目の係数(1-f)は反結合性
$\sigma_\text{g}$-d準位の非占有度であり、この準位の占有度が高ければ
高いほど相互作用の効果が減少するためにかけられている。
3項目は直交化に伴うパウリ反発(αは比例定数)を表す項である。

このように分子と金属dバンドの相互作用をを取り込むことで、従来の非摂動金属表面の性質のみに
依存した反応性理論(LDOS($\text{E}_\text{F}$)、$\text{N}_\text{h}$)に比べ、
反応性指標の向上が見込まれる。過去に、Hoffmannもまた分子と金属dバンドの相互作用について
指摘していた\cite{Hoffman1988}が、我々の指標では分子-金属sp相互作用後の準位を用いて
dバンドとの相互作用を考えているためより定量的な見積もりが可能になる。

\subsection{実際に指標を用いて各金属の反応性を比較する}
ここからは、式(\ref{deltaets})を用いて実際に金属の反応性を議論する。
そのためにそれぞれの系に対して$\varepsilon$、f、$\alpha$、$\text{V}^\text{2}$の値を求める
必要がある。

dバンドセンター$\varepsilon_d$は、状態密度(図\ref{fig:dosmetal})から得られる。
反結合性$\sigma_\text{g}$-d準位の占有度fは表面金属の局所的なd準位の占有度を用いて近似した。
金属sp軌道と相互作用した後の$\text{H}_\text{2}$の結合性、反結合性の準位の位置$\varepsilon _
{\sigma_\text{g}},\varepsilon _{\sigma_\text{u}^*}$は、金属にほどんと依存しないとして、
前者を-7eV,後者を1eVとした。
αはフィッティングパラメータとして、VはLMTO-ASA近似に基づいた式(\ref{v})より求めた。
\cite{Norskov1989}
\begin{eqnarray}
    \label{v}
    V = \eta \frac{ M_H M_d }{r^3}
\end{eqnarray}

$\text{M}_\text{H}$と$\text{M}_\text{d}$はそれぞれ水素や金属原子周りのポテンシャルであり、
$\eta$は結合角度に依存するパラメータ、rは原子間の距離である。
% このポテンシャルは電荷が一様に分布したJuliiumモデルを用いて計算できる

このようにして得られた種々の値から式(\ref{deltaets})より$\delta \text{E}_\text{d-hyb}$を求める。
計算した$\delta \text{E}_\text{d-hyb}$の値をエネルギー計算結果と共にプロットしたものが、
図\ref{fig:plot}である。
\begin{figure}[hbtp]
    \begin{center}
     \includegraphics[width=8cm]{figure/plot.png}
    \end{center}
    \caption{各金属上での吸着エネルギーと指標のプロット}
    \label{fig:plot}
\end{figure}
このプロットから、提案した反応性指標$\delta \text{E}_\text{d-hyb}$と実験結果には強い相関が
見て取れる。
\begin{table}[htb]
  \begin{center}
    \caption{式\ref{deltaets}の各項の計算値}
    \begin{tabular}{l|cccccc} \hline
    \label{table}
      Metal & $\varepsilon_\text{d}$ & $\text{V}^\text{2}$ & -2 $\frac{ V^2 }{\varepsilon _{\sigma_u^*}- \varepsilon _d}$ & $-2(1-f)\frac{V^2}{\varepsilon _d - \varepsilon _{\sigma_g}} $ & $\alpha \text{V}^\text{2}$ & $\delta \text{E}_\text{d-hyb}$ \\ \hline
      Cu                        & -2.67 & 2.42 & -1.32 & 0     & 1.02 & -0.30 \\
      Cu:$\text{Cu}_\text{3}$Pt & -2.35 & 2.42 & -1.44 & 0     & 1.02 & -0.42 \\
      Pt:$\text{Cu}_\text{3}$Pt & -2.55 & 9.44 & -5.32 & -0.42 & 3.96 & -1.78 \\
      Pt                        & -2.75 & 9.44 & -5.03 & -0.44 & 3.96 & -1.51 \\
      Ni                        & -1.48 & 2.81 & -2.27 & -0.10 & 1.18 & -1.19 \\ 
      Ni:NiAl                   & -1.91 & 2.81 & -1.93 & -0.11 & 1.18 & -0.86 \\
      Au                        & -3.91 & 8.10 & -3.30 & 0     & 3.40 & 0.10  \\ \hline
    \end{tabular}
  \end{center}
\end{table}

% \begin{figure}[hbtp]
%     \begin{center}
%      \includegraphics[width=14cm]{figure/table.png}
%     \end{center}
%     \caption{各項の計算値}
%     \label{fig:table}
% \end{figure}
式\ref{deltaets}の各項の値は表\ref{table}に示したとおりである。この表からパウリ反発の項である
$\alpha V^2$を1項目の$\varepsilon _{\sigma_u^*}$-d相互作用が打ち消すように働いて
いることが分かる。
これは、1項目と3項目におけるVの寄与が近いためである。

Auは反応性が著しく低い不活性な金属として知られており、反応性指標$\delta \text{E}_\text{d-hyb}$
(図\ref{fig:plot})からもその傾向が見て取れる。
これは、Auの$| \varepsilon _{\sigma_u^*} - \varepsilon _d|$が今考えているほかのどの
金属よりも大きいために、式(\ref{deltaets})の1項目$-2 \frac{ V^2 }{\varepsilon _
{\sigma_u^*} - \varepsilon _d}$が反発項$\alpha V^2$に対して小さくなってしまい
相互作用に伴う安定化がちょうどパウリ反発に打ち消されてしまうためである。この理由から、
Auはdバンドがあるにもかかわらず反応性の低い金属になっている。

\section{金属表面上でのCO吸着反応への適用}

ここまでは、金属表面上における水素分子の吸着反応でのエネルギーをその金属の反応性として
評価した。では、それ以外の反応においても$\delta \text{E}_\text{d-hyb}$は反応性の指標として
正しい結果を見積もることができるのであろうか。
そこでここからは、金属表面上における一酸化炭素の化学吸着反応について反応性を評価する。

\subsection{CO-金属間の相互作用}
金属dバンドと相互作用するCOの軌道は2$\pi^*$と5$\sigma$である。$\text{H}_\text{2}$
吸着の場合と同様に、d電子を持たない金属(Al)とのsp相互作用およびd電子を持つ金属(Pt)とのd相互作用
によってDOSがどのように変化するのかを図\ref{fig:codos}に示した。\cite{Hammer1996}
\begin{figure}[hbtp]
    \begin{center}
     \includegraphics[width=8cm]{figure/CO_dos.png}
    \end{center}
    \caption{CO-金属相互作用したDOS}
    \label{fig:codos}
\end{figure}

d相互作用によって反結合性5$\sigma$-d準位はフェルミ準位の上に、結合性2$\pi^*$-d準位は
フェルミ準位の下に位置している。したがってCOの吸着でも同じように、d相互作用はCO-金属間の結合を
安定化することが見て取れる。

\subsection{CO吸着反応での反応性指標}
続いて、式\ref{deltaets}を用いて指標の計算を行いたいところであるが、CO吸着においては若干の修正が
必要となる。なぜなら、実験や多くの理論的研究から空の2$\pi^*$軌道は占有された5$\sigma$軌道における
2倍の影響力をもつという結果が得られているためである。この事実を反映させるため、式\ref{deltaets}を
以下のように変形する。
\begin{equation}
    \begin{split}
        \label{ECO}
        E_{d-hyb} \simeq &-4\left[f\frac{V_\pi^2}{\varepsilon_{2\pi}-\varepsilon_d}-\alpha f V_\pi^2\right] \\
        &-2\left[(1-f)\frac{V_\sigma^2}{\varepsilon_d - \varepsilon_{5\sigma}}+-\alpha(1+f)V_\sigma^2\right]
    \end{split}
\end{equation}
この式を用いて$\text{E}_\text{d-hyb}$の計算を行っていこう。\\
dバンドの占有度fおよびdバンドセンター$\varepsilon_\text{d}$は$\text{H}_\text{2}$吸着の場合と
同様にDOSから見積もることができる。sp相互作用後の2$\pi$準位の位置$\varepsilon_{2\pi}$と
5$\sigma$準位の位置$\varepsilon_\text{5\sigma}$は表面の金属に依存しないためそれぞれ+2.5eV,-7eVと
する。結合強度$V_\pi$と$V_\sigma$はLMTO法に基づいて求めるため、$\text{H}_\text{2}$吸着の際に
用いた結合強度Vの定数倍で表すことができる。また、DFT計算より
$\text{V}_\sigma/\text{V}_\pi \simeq 1.3$という近似が成り立つことが分かったため、
\begin{eqnarray}
    \left\{
      \begin{array}{l}
        \text{V}_\sigma^2 \simeq (1.3)^2\beta\text{V}^2 \\
        \text{V}_\pi^2 \simeq \beta \text{V}^2
      \end{array}
    \right.
    \label{V_sigma_pi}
  \end{eqnarray}
と近似できる。ただし$\beta$は比例定数である。

以上で式\ref{ECO}の各項の計算ができるようになったため、実際に各金属について計算を行った。
結果は以下の表\ref{table2}ようになった。
% \begin{table}[htb]
%     \begin{center}
%       \caption{式\ref{deltaets}の各項の計算値}
%       \begin{tabular}{l|cccccc} \hline
%       \label{table}
%         Surface & $\varepsilon_\text{d}$ & f & $\text{V}^\text{2}$ & $\text{E}_\text{chem}$ & $\text{E}_\text{exp}$\\ \hline
%         Ni    & -2.67 & 2.42 & -1.32 & 0     & 1.02 & -0.30 \\
%         Ni/Ru & -2.35 & 2.42 & -1.44 & 0     & 1.02 & -0.42 \\
%         Ni@Cu & -2.55 & 9.44 & -5.32 & -0.42 & 3.96 & -1.78 \\
%         Cu    & -2.75 & 9.44 & -5.03 & -0.44 & 3.96 & -1.51 \\
%         Cu/Pt & -1.48 & 2.81 & -2.27 & -0.10 & 1.18 & -1.19 \\ 
%         $\text{Cu}_3$Pt                   & -1.91 & 2.81 & -1.93 & -0.11 & 1.18 & -0.86 \\
%         Au            & -3.91 & 8.10 & -3.30 & 0     & 3.40 & 0.10  \\ \hline
%       \end{tabular}
%     \end{center}
%   \end{table}
\begin{table}[hbtp]
    \begin{center}
    \caption{式\ref{ECO}の各項の計算値}
     \includegraphics[width=10cm]{figure/table2.png}
    \end{center}
    \label{table2}
\end{table}

\subsection{CO吸着エネルギーの計算}
これらの値を実際の反応の傾向と比較するため、各金属上におけるCO吸着エネルギーのDFT計算を行った。\\
交換相関汎関数にはGGAを用い、6層の金属fcc(111)面\footnote{Ruに対してはhcp(0001)を用いた}
から成るスラブの1面にCOを吸着させた。
40Ry(約544eV)のカットオフエネルギー\footnote{NiとCuに対しては50Ryのカットオフエネルギーを持つ
平面波基底を用いた}を持つ平面波基底を用い、すべての金属においてノルム保存型擬ポテンシャルを用いた。
$C_{3v}$と$C_{2v}$規約ブリルアンゾーンにそれぞれ6、15のk点を設定した。
以上の計算から得られたデータは、他の論文での計算結果や実験結果と定量的によい一致が見られた。

\subsection{CO吸着における実際の反応性と指標との相関}
求めた吸着エネルギーと式\ref{ECO}の値$\text{E}_\text{d-hyb}$とをプロットしたものが
以下の図\ref{fig:coplot}である。
\begin{figure}[hbtp]
    \begin{center}
     \includegraphics[width=10cm]{figure/plot2.png}
    \end{center}
    \caption{各金属上でのCO吸着エネルギーと指標のプロット}
    \label{fig:coplot}
\end{figure}

このプロットから、吸着質-金属d相互作用は金属上でのCO吸着の主要な反応性の傾向を見積もることが
できると言える。また、破線で示されている通り、単一結晶表面と複数結晶の構造とでは吸着エネルギー
が変化していることがわかる。これは、各金属においてdバンドセンターが違うために引き起こされる
変化であると考えられる。

\section{結論}
遷移金属、貴金属およびその合金において、反応性を理解するためには金属spバンドとの
相互作用によりシフトした後の分子軌道と、金属dバンドとの間の相互作用を考える必要がある。
また、反応性を決定づけている金属表面の重要なパラメータはdバンドセンター$\varepsilon_d$と
反結合性準位の占有度f、そして、結合エネルギーの要素Vである。

\bibliographystyle{unsrt}
\bibliography{MyCollection}

\end{document}

% Cu3Au no structure http://www.geocities.jp/ohba_lab_ob_page/structure6.html