\documentclass[12pt]{ltjsarticle}

\usepackage{graphicx}

\begin{document}
\begin{titlepage}
\title{金属表面の反応性を決定する指標}
\author{まるまる研 B4 \\ なまえ}
\date{2018/12/19}
\maketitle
%\thispagestyle{empty}
\tableofcontents

\end{titlepage}

\section{はじめに}
遷移金属を用いた触媒は広く使用されており、触媒材料探索の際には
遷移金属表面の化学的反応性が重要になる。
また、表面化学の分野においても金属表面の反応性を決定づけている物理的性質の理解については
長年興味がもたれており、分子-金属表面間での相互作用の理論的解釈について多くの研究が
なされてきた。 \\

\subsection{先行研究}

過去に考案されてきた代表的な反応性指標の2つは以下のとおりである。

\begin{itemize}
 \item LDOS($E_F$)
 \item d空孔の数 $N_h$
\end{itemize}

これらの指標について少し説明する。

\subsubsection{局所状態密度}
Peter,Hammanはフェルミ準位における局所状態密度(Local Density of States at Fermi level.
以下、LDOS($E_F$))が金属表面の反応性を決めていると提唱した。
これは、遷移金属のLDOS($E_F$)が小さければ小さいほど、その表面の反応性は低く、
逆にLDOS($E_F$)が大きければ反応性が高いという考え方である。\cite{PeterJ.Feibelman1984}

\subsubsection{d空孔の数}
金属のdバンドの上位の電子がsバンドに移ると空孔ができる。これをd空孔(d-holes)と呼び、
その数を$N_h$と表現する。

遷移金属のd空孔にs電子が戻るというd-s遷移(例えば5d空孔に6s軌道から遷移)により、パウリ反発
(分子軌道と金属d軌道とが重なることによるエネルギーの上昇)が減る。Pauli repulsion.pdf
これにより金属表面-分子間結合の形成が促されるため、結果的にd空孔の数$N_h$が多いほどその
金属の反応性が高くなるという結論が得られた。\cite{J.Harris1985}

\subsection{先行研究の破綻}
以上に述べた二つの指標はおおよその反応性を決定づけることができるもののすべての
遷移金属の反応性を満足するわけではなかった。\\
そこで今回は、あらゆる遷移金属表面の反応性を満たすような別の指標を提示する。

\section{計算}
遷移金属の反応性を見るため、いくつかの遷移金属表面上において水素の解離吸着反応に伴う
エネルギー変化を比較した。\\
実験に用いた遷移金属は以下の4つである。

\begin{itemize}
 \item Al
 \item Cu
 \item Pt
 \item $Cu_{3}$Pt
\end{itemize}

この4種類の金属が選出された理由はそれぞれ、dバンドを持たない金属、LDOS($E_F$)が大きい金属、
LDOS($E_F$)が小さい金属、そしてd空孔を持たない金属を表現するためである。

\subsection{計算方法}
各金属表面上での水素の解離吸着反応に伴うエネルギー変化を計算するにあたり、スラブモデルを用いて
DFT計算を行った。Al,Cu,Ptはfcc構造をとり格子定数はそれぞれ、3.96$\mbox{\AA}$、3.57$\mbox{\AA}$、3.93$\mbox{\AA}$である。
$Cu_3$Ptは$Cu_3$Auに似た、fccライクな構造をしており格子定数は3.68$\mbox{\AA}$である。
表面のスラブモデルはすべて、金属(111)6層+吸着真空層5層からなる2×2表面ユニットセルを持つ
スーパーセルを用いた。
50Ryのカットオフエネルギーを持つ平面波基底を用い、擬ポテンシャルとしては、AlにはBHS型、
CuとPtに対してはTroullier-Martins型のものを用いた。

\subsection{計算結果}
各金属の(111)面上でのH2のエネルギー計算結果は図\ref{fig:potential}のようになった。\\

\begin{figure}[hbtp]
    \begin{center}
     \includegraphics[width=8cm]{potential.png}
    \end{center}
    \caption{各金属上での吸着エネルギー}
    \label{fig:potential}
\end{figure}

金属表面から水素分子から表面までの距離Zを横軸にとり、縦軸がエネルギーである。
各金属の洗浄表面の全エネルギーと孤立水素分子のエネルギーの和を0としている。
水素吸着の定量的な議論のため、各金属とおして結合長bと高さZは同じ値を用いて計算を行った。
Al、Cu、$Cu_3$Pt内のCuサイトでの解離は大きなエネルギー障壁が確認できる一方で、
Ptや$Cu_3$Pt内のPtサイトには障壁はない。

我々の選択した反応経路の妥当性をを調べるため、水素分子をリラックスさせた場合の
エネルギーの計算も行った。計算結果は図\ref{fig:potential}の点線で示されているとおりである。
エネルギー障壁は若干小さくなるものの、傾向は変わらないことが確認できる。

また、計算したポテンシャルのこれらの特徴、すなわち、CuとAl表面では解離吸着に大きな障壁を伴い、
Ptと$Cu_3$Ptでは自発的に反応が進という傾向は実験の結果と同じであった。

\section{考察}

金属表面の反応性の議論を各金属の状態密度(Density of States)について考えることから始める。
図\ref{fig:dosmetal}はCu単体でのDOS, Pt単体でのDOS, その間に$Cu_3$PtのDOSを示している。

\begin{figure}[hbtp]
    \begin{center}
     \includegraphics[width=8cm]{figure/DosOfMetal.png}
    \end{center}
    \caption{各金属の電子状態密度}
    \label{fig:dosmetal}
\end{figure}

$Cu_3$Ptではdバンドの頂点がFermi準位と交差していることに注目してほしい。これは、$Cu_3$Ptの
dバンドがほどんど占有されておりd空孔がないことを示している。またCu3Pt中のPtサイトのLDOSは
Pt単体より小さいことが分かる。

以上の事実は従来の指標、LDOS($E_F$)や$N_h$を用いると$Cu_3$PtのPtサイトはPt単体よりも表面の
反応性は低いという結論に帰着する。しかしながら図\ref{fig:potential}によるとCu3PtのPtサイトはPt単体と同じ程度
の吸着エネルギー(反応性)になっている。したがってこれはLDOS($E_F$)や$N_h$では説明できない
ケースということになる。

\subsection{金属表面での吸着エネルギーの定式化}
吸着エネルギーを遷移金属のDOSと関連付けるため、Kohn-Sham方程式により求まる1電子軌道エネルギー
を用いる。
ここで、以下の近似を導入する。\\
2つの異なる金属表面上での吸着質のエネルギーを比較する際、我々は吸着質に近い場所の電子密度と
1電子ポテンシャルを”フリーズ”させることができ、異なる系に対して同じ密度とポテンシャルを
用いることができる。同様に、吸着質に近い場所より外側の残りの部分に対しても、密度とポテンシャル
を”フリーズ”でき、吸着質の存在とは独立に金属のみとして扱うことができる。
この近似を用いると異なる2つの金属表面にける吸着エネルギーの差は式(\ref{eabs})により表せる。

\begin{equation}
    \label{eabs}
    \delta E_{abs} = \int_{}^{E_F} \varepsilon n_{abs}(\varepsilon) d\varepsilon + 
    \delta E_{es}
\end{equation}

1項目は、吸着に関する1電子エネルギーの和の差であり、
2項目は、吸着質の静電エネルギーの差である。

変分原理のために、エネルギー差の誤差は密度とポテンシャルを”フリーズ”させていることによる
誤差の二次オーダーである。
式(\ref{eabs})より、大きな電荷移動がなく静電エネルギーの差が重要でない場合は障壁の大きさの
違いは1項目の電子エネルギーの和の差から見積もることができる。

\subsubsection{d電子を持たない金属表面の反応性}
式(\ref{eabs})を用いて吸着エネルギーの差における定性的な理解を試みよう。
まずは、d電子を持たず、s,p電子のみを持つ金属表面と水素分子の相互作用を考える。

\begin{figure}[hbtp]
    \begin{center}
     \includegraphics[width=11cm]{figure/interactedDOS.png}
    \end{center}
    \caption{分子と相互作用した金属の電子状態密度}
    \label{fig:interactedDOS}
\end{figure}

水素原子が金属表面のs電子、p電子とだけ相互作用した場合、
水素の結合性軌道と反結合性軌道の概略図は図\ref{fig:interactedDOS}A-Bのようになる。
反結合性軌道はFermi準位よりも少し上、結合性軌道はFermi準位よりも約7eV下に位置している。

\begin{figure}[hbtp]
    \begin{center}
     \includegraphics[width=11cm]{figure/dinteractedDOS.png}
    \end{center}
    \caption{分子-金属d相互作用した金属の電子状態密度}
    \label{fig:dinteractedDOS}
\end{figure}

今回の実験では、図\ref{fig:dinteractedDOS}のAl(111)がこの例である。
図\ref{fig:potential}のAl(111)のエネルギー曲線より、自由電子が解離吸着に対して大きな障壁を
もたらしていることを示している。\cite{Mysyrowicz1993}

\subsubsection{d電子をもつ金属の反応性}
次にAl以外の金属、つまりd電子を持つ金属の反応性について考える。\\
式(\ref{eabs})より、d電子の効果によるエネルギーの寄与は1電子エネルギーの差から
見積もることができる。

図\ref{fig:interactedDOS}B-Dはd準位の導入による効果を表している。
Cuのdバンドと水素の分子軌道の相互作用の結果、反結合性軌道のピークは分子準位の中心と
dバンド中心の上に、結合性軌道のピークは分子準位の中心とdバンド中心の下に位置したDOSが
得られている。

図\ref{fig:potential}より、dバンドとの相互作用がエネルギー障壁を下げるのに
寄与していることが分かる。

\subsection{dバンドと水素分子軌道の混成によるエネルギー変化の定式化}
この相互作用の大きさ、すなわちdバンドと水素分子軌道の混成によるエネルギーは
式(\ref{deltaepsilon})により決定する。

\begin{equation}
    \label{deltaepsilon}
    \Delta \varepsilon ~ \frac{ V^2 }{| \varepsilon _d - \varepsilon _ \sigma |}, \\
\end{equation}

これは3つの要因に分解できる。
\begin{enumerate}
  \item dバンドと相互作用する分子の結合性準位と反結合性準位の位置
  \item 分子軌道と金属dバンドとの間のハミルトニアン行列要素V
  \item 反結合性準位の占有度
\end{enumerate}

\subsubsection{Cuで見る分子ー金属d相互作用の効果}
図\ref{fig:interactedDOS}のCuを例にとってみると、反結合性準位がほとんど空なので、引力的である一方
結合性準位-d相互作用は、反結合性のものが占有されているため、分子と金属の重なりの直交化による
パウリ反発により反発的となる。トータルでは、分子ー金属d相互作用によってAlと比較してCuの方が
障壁が少し低いという結果になる。

\subsubsection{そのほかの金属で見る分子-d相互作用}
反結合性$ \sigma _u ^* -d$ 準位はどの金属においてもFermi準位の上に位置していることから
この準位は占有されていないことが分かる。したがって、$ \sigma _u ^*$相互作用はいつも
引力的である。
CuとCuPt3では結合強度(V)とdバンドの位置がどちらもほとんど同じであるため、$ \sigma _u ^*$
相互作用の大きさは同程度であると考えられる。
PtとCuではdバンドの位置がほとんど同じ(図\ref{fig:dosmetal})であるが、
結合強度がPtの方が大きいためPtはCuよりも$ \sigma _u ^*$相互作用がより大きい。
結合強度の違いはCuは3d、Ptはよりバンドが大きい5dであることに起因する。

$\sigma _g$相互作用は、表面の金属によってかなり大きさが変わる。

図\ref{fig:dinteractedDOS}からわかるように、反結合性$\sigma _g -d$ピークはCuでは
ちょうどdバンドの上あたりにあり
Cu3Ptではフェルミ準位のあたり、Ptではフェルミ準位の上あたりにシフトしていく。

\section{反応性の新たな指標}
より定量的な議論のため、反応性の指標を導入する。\\
dバンドの相互作用によるエネルギー差は式(\ref{deltaepsilon})を用いて式(\ref{deltaets})と
表せる。

\begin{eqnarray}
    \label{deltaets}
    \delta E_{ts} = -2 \frac{ V^2 }{\varepsilon _{\sigma_u^*} - \varepsilon _d}
                    -2(1-f)\frac{V^2}{\varepsilon _d - \varepsilon _{\sigma_g}}
                    + \alpha V^2
\end{eqnarray}

1項目は$\sigma _u ^* -d$相互作用によるエネルギーであり、
2項目は$\sigma _g -d$相互作用によるエネルギーであり、
3項目は直交化に伴うパウリ反発(αは比例定数)である。
係数の2は上下のスピンに対応しており、(1-f)は非占有度合である。

このように、分子と金属dバンドの相互作用をを取り込むことで、従来の非摂動金属表面の性質のみに
依存した反応性理論に比べ、反応性指標の向上が見込まれる。

過去に、Hoffmannもまた分子と金属dバンドの相互作用について指摘していたが、我々の指標では
分子-金属sp相互作用後の準位を用いてdバンドとの相互作用を考えているためより定量的な見積もりが
可能になった。

\subsection{実際に指標を用いて各金属の反応性を比較する}
ここからは、式(\ref{deltaets})を用いて実際に金属の反応性を議論する。
そのためにそれぞれの系に対して$\varepsilon$、f、$\alpha$、$V^2$の値を求める必要がある。

dバンドセンター$\varepsilon_d$は、図\ref{fig:dosmetal}からわかる。\\
反結合性$\varepsilon _{\sigma_g}$の占有度fは表面のd状態の占有度を用いて近似した。
金属sp軌道と相互作用した後のH2の結合性、反結合性の準位の位置$\varepsilon _{\sigma_g},
\varepsilon _{\sigma_u^*}$は、金属にほどんと依存しないとして、
前者を-7eV,後者を1eVとした。
αはフィッティングパラメータとして、Vは式(\ref{v})(\cite{Nørskov1989})より、求めた。

\begin{eqnarray}
    \label{v}
    v = \eta \frac{ M_H M_d }{r^3}
\end{eqnarray}

$M_H$や$M_d$は水素や金属原子周りのポテンシャル(Jellium計算で得られる)であり、
$\eta$は結合角度に依存するパラメータである。

このようにして得られた種々の値から式(\ref{deltaets})より$\delta E_{ts}$を求める。
計算した$\delta E_{ts}$の値をエネルギー計算結果と共にプロットしたものが、
図\ref{fig:plot}である。

\begin{figure}[hbtp]
    \begin{center}
     \includegraphics[width=8cm]{figure/plot.png}
    \end{center}
    \caption{各金属上での吸着エネルギーと指標のプロット}
    \label{fig:plot}
\end{figure}

このプロットから、提案した反応性指標$\delta E_{ts}$と実験結果には強い相関が見て取れる。

\begin{figure}[hbtp]
    \begin{center}
     \includegraphics[width=14cm]{figure/table.png}
    \end{center}
    \caption{各項の計算値}
    \label{fig:table}
\end{figure}

各項の値は表\ref{fig:table}に示したとおりである。この表からパウリ反発の項である
$\alpha V^2$を1項目の$\varepsilon _{\sigma_u^*}$相互作用が打ち消すように働いて
いることが分かる。
これは、1項目と3項目におけるVのスケーリング、寄与が近いためである。

Auは反応性が著しく低い不活性な金属として知られており、反応性指標$\delta E_{ts}$からも
その傾向が見て取れる。(図\ref{fig:plot})
これは、Auの$| \varepsilon _{\sigma_u^*} - \varepsilon _d|$が今考えているほかのどの
金属よりも大きいために、式(\ref{deltaets})の1項目$-2 \frac{ V^2 }{\varepsilon _
{\sigma_u^*} - \varepsilon _d}$が反発項$\alpha V^2$に対して小さくなってしまい
相互作用に伴う安定化が反発にちょうど打ち消されてしまうためである。この理由から、
Auはdバンドがあるにもかかわらず反応性の低い金属になっている。

\section{結論}
遷移金属、貴金属およびその合金において、水素の反応性傾向を理解するためには、金属の
自由電子との相互作用によりシフトした後の分子軌道と、金属dバンドとの間の相互作用を
考えなければならない。また、反応性と決定づけている表面の重要なパラメータはdバンド
の位置$\varepsilon_d$と反結合性準位の占有度f、そして、結合エネルギーの要素Vである。

\bibliographystyle{unsrt}
\bibliography{MyCollection}
\end{document}